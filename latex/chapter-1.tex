\documentclass{article}

\usepackage{hyperref}
\begin{document}

\title{About}

\maketitle


A web service for easy knowledge access of IPCC report for use in climate plans and their supporting cultural and educational projects


\emph{Proof of concept prototype}


A project of the open research group semanticClimate — \emph{‘liberating knowledge from climate-related reports’}.


July 2023. 

\begin{itemize}
\item Web: \href{https://semanticclimate.org/}{https://semanticclimate.org/} 


\item Discussion and specifications: \href{https://github.com/petermr/semanticClimate/discussions/32}{GitHub} 


\item Code: \href{https://github.com/petermr/semanticClimate}{semanticClimate} 


\item Contact: Simon Worthington, \href{mailto:simon.worthington@tib.eu}{simon.worthington@tib.eu} 


\end{itemize}

semanticClimate is led by young Indian scientists and supported by The National Institute of Plant Genome Research (NIPGR) – Delhi, and partners: Open Science Lab, TIB.


\subsection{Project summary}\label{H2966887}



City and regional plans are important tools to help mitigate the impacts of climate change. 


The project is a prototype web service that allows users to search and collate IPCC Report information related to city climate plans. Using the service users can easily compile content from the IPCC reports as referenced readers.


Currently IPCC reports are not supported by search services that allow for granular indexing. The semanticClimate uses Linked Open Data (LOD) and Wikidata / Wikibase technologies to enable better visibility for the IPCC Report contents.


The IPCC Reports remain as the authoritative referenced source content, with the semanticClimate acting as a search and reuse layer.


The goal of the project is to help produce better informed and communicated city climate plans to produce more effective and democratically supported climate outcomes.


\subsubsection{Use case: Supporting city climate plan authors}\label{H9688122}



Climate plan authors need to find IPCC Report recommendations, sections, visualisations, and external references and use these in their own city climate plans as well as to distribute the content to their community for the purpose of making them democratically — understandable, accountable, and transparent.


Using the semanticClimate’s research instrument called the Knowledge Explorer an author could create a custom reader to share with stakeholders as an automatically typeset content package — including external citations — with linked references to the original source IPCC Reports.


semanticClimate works by breaking down the IPCC Reports into the smallest parts possible, for example - sentences, acronyms, section headers, images, citations — and semantically tags each part. This tagging then allows machine readability and use of machine learning tooling, which in effect unlocks the content for findability and recombinations as search results, knowledge graphs, and in addition as full-text content which can be reused as collated and automatically typeset readers, or new packages like a glossary list that hyperlinks references all original sections of the Reports where they are used.


\subsection{City climate plans}\label{H3319789}



\subsubsection{Context for city climate plans (videos): }\label{H2408854}


\begin{itemize}
\item Munich Security Conference, \emph{Wider mitigation measures} (2023) “\href{https://www.youtube.com/live/wbLB6TwLIto?feature=share}{To Greener Pastures: Advancing Joint Climate Action}”;


\item Munich Security Conference: (2022) "\href{https://www.youtube.com/live/2FuPxnFMdG0?feature=share&t=3128}{The Role of Cities: Democratic Game Changers?}".


\end{itemize}

\subsubsection{Example city climate plans: }\label{H5715858}


\begin{itemize}
\item \href{https://moef.gov.in/wp-content/uploads/2017/08/Delhi-State-Action-Plan-on-Cimate-Change.pdf}{Delhi State Action Plan on Climate Change} (2017)


\item \href{https://mcap.mcgm.gov.in/}{Mumbai}


\item Cambridge


\item Hannover


\item \href{https://www.berlin.de/sen/uvk/en/climate-action/berlin-energy-and-climate-protection-programme-2030-bek-2030/}{Berlin Energy and Climate Protection Programme 2030}


\item Frankfurt


\end{itemize}

\subsubsection{Plan support programmes :}\label{H3402659}


\begin{itemize}
\item Cultural  - The Floating University (Berlin), \href{https://floating-berlin.org/}{https://floating-berlin.org/}


\item Informational


\item Economic - \emph{Mission metrics: }\emph{\href{https://www.ucl.ac.uk/bartlett/public-purpose/publications/2023/feb/mission-metrics-policy-evaluation-tools-cities-optimise-learning-green}{Policy evaluation tools for cities to optimise learning for the green transition}} (2023), UCL Institute for Innovation and Public Purpose (IIPP).


\item Innovations and tech


\item Health


\item Participators


\item Democratic, accountable, and measurable - 


\item Modelling


\end{itemize}
\begin{itemize}
\item \href{https://plan.lamayor.org/}{Los Angeles’ Green New Deal} (2019)


\end{itemize}

\subsection{Work programme}\label{H9711914}



semanticClimate research and development work creates open source software to enable what we call a ClimateExplorer and on top of this we run hackathons as LearnByExploring events.


semanticClimate code link: \href{https://github.com/petermr/semanticClimate}{https://github.com/petermr/semanticClimate} 

\begin{itemize}
\item Proof of concept prototype with: TIB – NextGenBooks/COPIM


\item Hackathon with FSCI Summer School and UCLA Library, 2023 (In Cooperation with class module - \href{https://force11.org/fsci/post/course-list-with-abstracts-2023/#e08}{E08} – Publishing from Collections Using Linked Open Data Source and Computational Publishing Pipelines) 

\begin{itemize}
\item Prototype mockup


\item Prototype AI mockup


\end{itemize}

\item Berlin Hackathon (TIB) (Oct. ‘23) 


\item Seek funding for further prototyping development rounds and hackathons 


\end{itemize}
\end{document}
